\documentclass[11pt, a4paper]{article}
\usepackage[utf8x]{inputenc}
\usepackage[swedish, english]{babel}		% last is active
%\usepackage{graphicx}
\usepackage{amsmath}					% to be able to \split eqs
\usepackage{amssymb}					% Real/Imaginary fonts
\usepackage{units}
\usepackage[tight, hang]{subfigure}
\usepackage{url}
\usepackage{tikz}						% for drawing
\usetikzlibrary{shapes, arrows, decorations.markings, decorations.pathmorphing, decorations.pathreplacing, calc}
\usepackage{fancyhdr}
\usepackage{float}						% H-positioned and custom floats
\usepackage{datetime}					% to fix date format
\usepackage[usenames,dvipsnames]{pstricks}
%\usepackage{epsfig}
%\usepackage{pst-grad} % For gradients
%\usepackage{pst-plot} % For axes
\usepackage{pgfplots}
% =================== some local stuff ================= %
\newcommand{\degree}{\ensuremath{^\circ}}
\newcommand{\todayswe}{\the\year-\twodigit\month-\twodigit\day}


\def\contacts{Torbjørn Ludvigsen, tolu0022@student.umu.se\\Olof Lenti, olle0004@student.umu.se\\
Yunus Gures, yunusgures@gmail.com}
\def\names{Torbjørn Ludvigsen, Olof Lenti, Yunus Gures}
\def\dept{Department of Physics}
\def\course{Non invasive measurement theory}
\def\lab{Optical measurements:\\Determination of the Damping of a Pendulum with Time of Flight}
\def\supervisors{Patrick Ehlers\\Isak Silander\\ Amir Khodabakhsh}
\date{\todayswe}
% custom commands
\newcommand\OpVec[1]{\boldsymbol{\hat{#1}}}		% bold with hat for operator vectors
%\newcommand\Sup[1]{\textsuperscript{\tiny{#1}}}		% 1st, 2nd.. and so on
\newenvironment{eqn}{\begin{equation*} \begin{split}}{\end{equation*} \end{split}}
% header

% document
\begin{document}
\pagestyle{fancy}
\begin{titlepage}
	\begin{center}
		\course\\
		\Large{\lab}\vspace{2mm}
		\hrule\vspace{2mm}
		\tiny{\contacts}\vspace{2mm}
		\hrule
	\end{center}
	\vspace{4mm}

	\begin{abstract}

  $\alpha_{alu} =\unit[(23.0 \pm 0.1)\cdot10^{-6}]{K^{-1}}$ 

  $\alpha_{sst} = \unit[(15.8 \pm 0.2)\cdot10^{-6}]{K^{-1}}$, 
    which is only 1 \% off tabulated values \cite{ph, thex}.

	\end{abstract}
	\vfill
	\hrule\vspace{2mm}
	\centering
		\tiny{Supervisor: \supervisors}
	%\end{center}
\end{titlepage}

\pagestyle{plain}
\vspace{2cm}
\section{Introduction}

\section{Theory}
\subsection{Optics}
\subsection{Circuits}
\subsection{maths}

blabla



By beginning with the equation
\[
v_{max} = Ce^{at} + De^{bt}.
\]
By breaking out $e^{at}$ and taking the logarithm we end up with the equation
\[
\ln(v_{max}) = \ln(C + De^{\frac{b}{a}t}) + at
\]
For small $t$ the first term will be nearly constant. A linear fit can be made to find the slope $a$.
In a similar fashion we can break out $e^{bt}$

\section{Experimental Setup}
\section{Procedure}
\section{Error calculations}
\section{Results}
\section{Discussion}
\section{Summary and Conclusions}
\vfill

\begin{thebibliography}{99}
	\bibitem{ph} Nordling, C., Österman, J. (2006). 
  \textit{Physics Handbook  $8^{th}$}\\
  Lund, Sweden, Studentlitteratur.
\end{thebibliography}

\begin{appendix}
\end{appendix}

% nY KOMMENTAR

\end{document}
%\begin{figure}[H]
%	\centering
%	\begin{tikzpicture}[scale = 0.95]
%		\def\mr{1.5};
%		\draw (0,0) ellipse (0.5 and \mr);
%		\begin{scope}
%			\clip (0,0) ellipse (0.5 and \mr);
%			\foreach \i in {0, 1, ..., 15} {
%				\draw (-0.5, {-3 + 0.25*\i}) -- (0.5, {-1 + 0.25 * \i});
%			}
%		\end{scope}
%		\path[fill = white, opacity = 0.7] (0, 0) circle (0.2);
%		\node at (0, 0) {$S$};
%		\begin{scope}
%			% overlapping regions will cancel out
%			\draw[fill = white] (5, 0) circle (3);
%		\end{scope}
%		\draw (0, \mr) -- ++(2.8, 0) to[out = 0, in = -135] ++(0.3, 0.3)
%			  (0, -\mr) -- ++(2.8, 0) to[out = 0, in = 135] ++(0.3, -0.3);
%		\node at (5, 0) {$V$};
%	\end{tikzpicture}
%	\caption{Helmholtz resonator}
%	\label{f:helmholtz}
%\end{figure}
